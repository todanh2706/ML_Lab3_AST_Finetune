% \pagebreak
% \section{Đánh giá và phân tích kết quả}

% \subsection{Kết quả định lượng trên UrbanSound8K}

% Sau khi huấn luyện và lưu checkpoint cho từng fold, nhóm tiến hành đánh giá lại
% toàn bộ 10 mô hình trên tập kiểm thử tương ứng của mỗi fold. Quá trình đánh giá được
% thực hiện riêng cho hai cấu hình:

% \begin{itemize}
%     \item AST-P: backbone khởi tạo từ checkpoint đã fine-tune trên AudioSet,
%     sau đó thay đầu ra 10 lớp và fine-tune trên UrbanSound8K.
%     \item AST-S: backbone khởi tạo từ mô hình tiền huấn luyện trên ImageNet,
%     chỉ fine-tune trực tiếp trên UrbanSound8K, không sử dụng checkpoint AudioSet.
% \end{itemize}

% Bảng dưới đây tóm tắt kết quả trung bình trên 10 fold:

% \begin{table}[h]
%     \centering
%     \begin{tabular}{lccc}
%         \hline
%         Cấu hình mô hình & Mean accuracy (\%) & Std (\%) & Micro accuracy (\%) \\
%         \hline
%         AST-S  & 79.75 & 3.76 & 79.70 \\
%         AST-P  & 86.31 & 4.20 & 86.27 \\
%         \hline
%     \end{tabular}
%     \caption{Kết quả 10-fold cross-validation trên UrbanSound8K cho hai cấu hình AST.}
%     \label{tab:us8k_results}
% \end{table}

% Đối với cấu hình AST-P, độ chính xác trên từng fold dao động từ khoảng
% 76.86\% (fold khó nhất) đến 92.12\% (fold tốt nhất), với giá trị trung bình
% 86.31\% và độ lệch chuẩn 4.20\%. Micro accuracy tính trực tiếp từ confusion matrix
% tổng hợp trên toàn bộ dữ liệu là 86.27\%, gần trùng với giá trị trung bình theo fold,
% cho thấy việc chia fold và tổng hợp là nhất quán.

% Trong khi đó, cấu hình AST-S đạt độ chính xác trung bình 79.75\% với độ lệch
% chuẩn 3.76\%. Các fold dao động trong khoảng từ 72.65\% đến 84.19\%, micro accuracy
% tính từ confusion matrix tổng là 79.70\%. Như vậy, so với AST-S, mô hình
% AST-P cải thiện khoảng 6.5 điểm phần trăm về độ chính xác trung bình trên
% UrbanSound8K.


% \subsection{Phân tích confusion matrix}

% Để quan sát chi tiết hơn cách mô hình nhầm lẫn giữa các lớp, nhóm tổng hợp
% confusion matrix trên toàn bộ 10 fold cho từng cấu hình. Thứ tự các lớp trong
% bảng là:
% air\_conditioner, car\_horn, children\_playing, dog\_bark,
% drilling, engine\_idling, gun\_shot, jackhammer, siren, street\_music.

% \begin{table}[h]
%     \centering
%     \scriptsize
%     \setlength{\tabcolsep}{3pt}
%     \begin{tabular}{lrrrrrrrrrr}
%         \hline
%         Thật \textbackslash\ Dự đoán
%         & air\_cond. & car\_horn & child\_play & dog\_bark
%         & drilling & eng\_idle & gun\_shot & jackhammer & siren & street\_music \\
%         \hline
%         air\_cond.    & 744 &   0 &  26 &  29 &  86 &  46 &   0 &   9 &  11 &  49 \\
%         car\_horn     &   0 & 399 &   3 &   2 &   1 &   1 &   0 &  13 &   0 &  10 \\
%         child\_play   &   2 &   0 & 951 &   6 &  11 &   0 &   0 &   0 &   6 &  24 \\
%         dog\_bark     &  18 &   2 &   9 & 958 &   3 &   0 &   0 &   0 &   5 &   5 \\
%         drilling      &  27 &  11 &   8 &   4 & 856 &  14 &   0 &  58 &   4 &  18 \\
%         eng\_idle     & 102 &   3 &   5 &   0 &  40 & 715 &   0 & 118 &   4 &  13 \\
%         gun\_shot     &   0 &   0 &   0 &   4 &   1 &   0 & 368 &   1 &   0 &   0 \\
%         jackhammer    &  22 &   0 &   1 &   0 & 171 &  10 &   0 & 782 &   0 &  14 \\
%         siren         &  12 &  11 &   5 &  43 &   1 &   0 &   0 &   1 & 849 &   7 \\
%         street\_music &   9 &   3 &  52 &   3 &   8 &   4 &   0 &   5 &   5 & 911 \\
%         \hline
%     \end{tabular}
%     \caption{Confusion matrix tổng hợp của cấu hình AST-P trên UrbanSound8K.}
%     \label{tab:cm_astp}
% \end{table}

% \begin{table}[h]
%     \centering
%     \scriptsize
%     \setlength{\tabcolsep}{3pt}
%     \begin{tabular}{lrrrrrrrrrr}
%         \hline
%         Thật \textbackslash\ Dự đoán
%         & air\_cond. & car\_horn & child\_play & dog\_bark
%         & drilling & eng\_idle & gun\_shot & jackhammer & siren & street\_music \\
%         \hline
%         air\_cond.    & 556 &   2 &  33 &  97 &  77 &  78 &   0 &  33 &  38 &  86 \\
%         car\_horn     &   1 & 392 &   2 &   0 &   7 &   0 &   0 &  10 &   0 &  17 \\
%         child\_play   &   8 &   0 & 879 &  48 &   6 &   3 &   0 &   1 &   6 &  49 \\
%         dog\_bark     &  10 &   2 &  31 & 923 &   8 &   6 &   3 &   0 &   7 &  10 \\
%         drilling      &  36 &  16 &  12 &  18 & 774 &  20 &   2 &  70 &  28 &  24 \\
%         eng\_idle     & 114 &   1 &  17 &  10 &  14 & 680 &   2 & 121 &  28 &  13 \\
%         gun\_shot     &   0 &   0 &   0 &   5 &   0 &   0 & 369 &   0 &   0 &   0 \\
%         jackhammer    &  55 &   3 &   4 &   0 & 182 &  53 &   4 & 688 &   0 &  11 \\
%         siren         &   9 &   3 &  25 &  20 &  32 &   4 &   0 &   0 & 811 &  25 \\
%         street\_music &  20 &  13 &  42 &  13 &   6 &   7 &   0 &   5 &   7 & 887 \\
%         \hline
%     \end{tabular}
%     \caption{Confusion matrix tổng hợp của cấu hình AST-S trên UrbanSound8K.}
%     \label{tab:cm_asts}
% \end{table}

% Từ hai confusion matrix trong Bảng~\ref{tab:cm_astp} và Bảng~\ref{tab:cm_asts},
% nhóm rút ra một số nhận xét định tính về hành vi của mô hình.

% \paragraph{Các lớp dễ phân biệt.}
% Các lớp có tính chất âm thanh rõ ràng, mang tính sự kiện ngắn và đặc trưng như
% gun\_shot, car\_horn, dog\_bark, children\_playing thường đạt độ chính xác rất cao
% ở cả hai cấu hình. Ví dụ, lớp gun\_shot hầu như luôn được dự đoán đúng, số lượng
% mẫu bị nhầm sang lớp khác là rất ít. Điều này phù hợp với trực giác khi đây là những
% âm thanh có biên dạng thời gian và phổ tần tương đối đặc trưng.

% \paragraph{Các lớp khó, dễ gây nhầm lẫn.}
% Những lớp mang tính nền tiếng ồn đô thị và tiếng máy móc như air\_conditioner,
% engine\_idling, drilling, jackhammer và siren có xu hướng bị nhầm lẫn lẫn nhau.
% Trong confusion matrix của cả hai cấu hình, các hàng tương ứng với các lớp này
% có số lượng mẫu bị dự đoán sang nhau khá lớn. Điều này phản ánh việc đặc trưng
% log-Mel của các loại tiếng ồn công nghiệp có phổ tần chồng lấn và hình thái
% thời gian tương đối giống nhau.

% So sánh hai confusion matrix cho thấy khi sử dụng pretraining AudioSet, số lượng
% nhầm lẫn giữa một số cặp lớp giảm xuống, đặc biệt là đối với các lớp có tính chất
% âm thanh gần với các nhãn trong AudioSet. Điều này gợi ý rằng kiến thức học được
% từ AudioSet giúp backbone AST phân tách tốt hơn các kiểu tiếng ồn phức tạp,
% dù vẫn còn một mức độ nhầm lẫn nhất định giữa các lớp có bản chất rất giống nhau.

% \subsection{So sánh AST-S và AST-P}

% Kết quả trong Bảng~\ref{tab:us8k_results} cho thấy:

% \begin{itemize}
%     \item Pretraining trên AudioSet mang lại mức cải thiện đáng kể về hiệu năng:
%     từ khoảng 79.75\% (AST-S) lên 86.31\% (AST-P), tương đương
%     tăng khoảng 6.5 điểm phần trăm trên UrbanSound8K.
%     \item Độ lệch chuẩn giữa các fold của hai cấu hình đều ở mức khoảng 4\%, cho thấy
%     hiệu năng ổn định trên các fold khác nhau, mặc dù một số fold vẫn khó hơn
%     do phân bố lớp và điều kiện thu âm.
%     \item Khoảng cách giữa độ chính xác huấn luyện (gần 100\% ở các epoch cuối)
%     và độ chính xác kiểm thử (khoảng 80–86\%) cho thấy mô hình vẫn có hiện tượng
%     overfitting ở cả hai cấu hình. Tuy nhiên, với cùng một pipeline huấn luyện,
%     việc sử dụng trọng số tiền huấn luyện từ AudioSet giúp cải thiện đáng kể khả năng
%     tổng quát hóa trên tập kiểm thử.
% \end{itemize}

% Nhìn chung, kết quả thực nghiệm trên UrbanSound8K phù hợp với kết luận của tác giả
% AST: pretraining trên một tập dữ liệu âm thanh lớn như AudioSet mang lại lợi ích rõ rệt
% cho các bài toán phân loại âm thanh môi trường trên những tập dữ liệu quy mô vừa và nhỏ.
% Trong bối cảnh bài toán này, AST-P không chỉ đạt độ chính xác cao hơn mà còn
% giữ được độ ổn định tốt trên 10 fold, thể hiện vai trò quan trọng của pretraining
% đúng miền dữ liệu so với chỉ pretraining trên ImageNet.

\pagebreak
\section{Đánh giá và phân tích kết quả}

\subsection{Kết quả định lượng trên UrbanSound8K}

Sau khi huấn luyện và lưu checkpoint cho từng fold, nhóm tiến hành đánh giá lại
toàn bộ 10 mô hình trên tập kiểm thử tương ứng của mỗi fold. Quá trình đánh giá được
thực hiện riêng cho hai cấu hình:

\begin{itemize}
    \item AST-P: backbone khởi tạo từ checkpoint đã fine-tune trên AudioSet,
    sau đó thay đầu ra 10 lớp và fine-tune trên UrbanSound8K.
    \item AST-S: backbone khởi tạo từ mô hình tiền huấn luyện trên ImageNet,
    chỉ fine-tune trực tiếp trên UrbanSound8K, không sử dụng checkpoint AudioSet.
\end{itemize}

Bảng dưới đây tóm tắt kết quả trung bình trên 10 fold:

\begin{table}[h]
    \centering
    \begin{tabular}{lccc}
        \hline
        Cấu hình mô hình & Mean accuracy (\%) & Std (\%) & Micro accuracy (\%) \\
        \hline
        AST-S  & 74.41 & 5.72 & 74.41 \\
        AST-P  & 80.08 & 6.76 & 80.08 \\
        \hline
    \end{tabular}
    \caption{Kết quả 10-fold cross-validation trên UrbanSound8K cho hai cấu hình AST.}
    \label{tab:us8k_results}
\end{table}

Đối với cấu hình AST-P, độ chính xác trên từng fold dao động từ khoảng
65.62\% (fold khó nhất) đến 90.32\% (fold tốt nhất), với giá trị trung bình
80.08\% và độ lệch chuẩn 6.76\%. Micro accuracy tính trực tiếp từ confusion matrix
tổng hợp trên toàn bộ dữ liệu cũng đạt 80.08\%, trùng khớp với giá trị trung bình theo
fold, cho thấy quá trình tổng hợp và đánh giá là nhất quán.

Trong khi đó, cấu hình AST-S đạt độ chính xác trung bình 74.41\% với độ lệch
chuẩn 5.72\%. Các fold dao động trong khoảng từ 65.95\% đến 83.33\%, micro accuracy
tính từ confusion matrix tổng là 74.41\%. Như vậy, so với AST-S, mô hình
AST-P cải thiện khoảng 5.7 điểm phần trăm về độ chính xác trung bình trên
UrbanSound8K.


\subsection{Phân tích confusion matrix}

Để quan sát chi tiết hơn cách mô hình nhầm lẫn giữa các lớp, nhóm tổng hợp
confusion matrix trên toàn bộ 10 fold cho từng cấu hình. Thứ tự các lớp trong
bảng là:
air\_conditioner, car\_horn, children\_playing, dog\_bark,
drilling, engine\_idling, gun\_shot, jackhammer, siren, street\_music.
\begin{figure}[H]
    \centering
    \includegraphics[width=0.75\linewidth]{img/confusion_matrix.png}
    \caption{Ma trận nhầm lẫn thể hiện khả năng dự đoán của 2 cấu hình}
    \label{fig:conf_matrix}
\end{figure}
Từ hai confusion matrix trong Hình~\ref{fig:conf_matrix},
nhóm rút ra một số nhận xét định tính về hành vi của mô hình.

\paragraph{Các lớp dễ phân biệt.}
Các lớp mang tính sự kiện ngắn, có đặc trưng phổ tần rõ ràng như
gun\_shot, car\_horn, dog\_bark và children\_playing đạt độ chính xác cao
ở cả hai cấu hình. Đặc biệt, lớp gun\_shot gần như không bị nhầm lẫn sang
các lớp khác, phản ánh khả năng học tốt các mẫu âm thanh có biên dạng
thời gian và phổ tần đặc trưng.

\paragraph{Các lớp khó, dễ gây nhầm lẫn.}
Những lớp mang tính nền tiếng ồn đô thị và tiếng máy móc như
air\_conditioner, engine\_idling, drilling, jackhammer và siren
có xu hướng bị nhầm lẫn lẫn nhau. Điều này thể hiện rõ qua các giá trị
ngoài đường chéo chính trong confusion matrix, đặc biệt giữa
engine\_idling và jackhammer, cũng như giữa air\_conditioner và
street\_music.

So sánh hai cấu hình cho thấy với AST-P, mức độ nhầm lẫn giữa các lớp
tiếng ồn công nghiệp nhìn chung giảm so với AST-S, đặc biệt ở các lớp
liên quan đến máy móc và động cơ. Điều này gợi ý rằng kiến thức học được
từ AudioSet giúp backbone AST trích xuất đặc trưng âm thanh tốt hơn
trong các bối cảnh phức tạp, dù vẫn còn tồn tại nhầm lẫn giữa các lớp
có bản chất âm thanh rất gần nhau.

\subsection{So sánh AST-S và AST-P}

Từ kết quả trong Bảng~\ref{tab:us8k_results}, có thể rút ra các kết luận sau:

\begin{itemize}
    \item Pretraining trên AudioSet giúp cải thiện đáng kể hiệu năng mô hình,
    từ 74.41\% (AST-S) lên 80.08\% (AST-P), tương đương mức tăng khoảng
    5.7 điểm phần trăm trên UrbanSound8K.
    \item Độ lệch chuẩn giữa các fold của AST-P cao hơn AST-S, phản ánh sự
    khác biệt về độ khó giữa các fold, nhưng vẫn cho thấy xu hướng cải thiện
    nhất quán khi sử dụng pretraining AudioSet.
    \item So sánh giữa độ chính xác huấn luyện (khoảng 81--87\%) và độ chính xác
    kiểm thử (khoảng 74--80\%) cho thấy cả hai cấu hình đều tồn tại hiện tượng
    overfitting. Tuy nhiên, với cùng pipeline huấn luyện, AST-P thể hiện
    khả năng tổng quát hóa tốt hơn so với AST-S.
\end{itemize}

Nhìn chung, kết quả thực nghiệm trên UrbanSound8K tiếp tục khẳng định
vai trò quan trọng của pretraining trên tập dữ liệu âm thanh quy mô lớn
như AudioSet. Trong bối cảnh bài toán này, AST-P không chỉ đạt độ chính xác
cao hơn mà còn cải thiện khả năng phân biệt các lớp âm thanh phức tạp,
cho thấy lợi thế rõ rệt của pretraining đúng miền dữ liệu so với chỉ
sử dụng pretraining trên ImageNet.